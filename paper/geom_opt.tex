\documentclass{amsart}

\usepackage[margin=1in]{geometry}
\usepackage{lipsum}

\title{Shape optimization for tracer distribution in Poiseuille flow}

\author{Manuchehr Aminian \\
California State Polytechnic University, Pomona \\ {maminian@cpp.edu}}



\begin{document}

\maketitle

\section{Motivation}
\begin{itemize}
\item Taylor dispersion applies to circular cross sections
\item Aris papers establish moment equations and solvability (?)
\item Chatwin studies cross section dependence of enhanced diffusivity 
in rectangles
\item George paper observes differing skewness in parallel plate 
and circular cross sections; what gives
\item Stone (?) and others explore cross section dependence and 
use shallow-water-like methods to reduce equations and arrive 
at approximate statements for some piecewise-defined domains
\item Our previous papers establish 
\begin{enumerate}
\item rectangular dependence and 
purely ballistic induced skewness with plug initial condition; (PRL)
\item Short and long time asymptotics; agreement with experiment; 
extension to ellipses (Science)
\item Further exploration of cross-section dependence by heuristic 
``racetrack" approach (SAPM)
\end{enumerate}
\end{itemize}

Given all these, my interest now is in a purely numerical exploration 
of cross section dependence which culminates from our previous work. 
I would like to arrive at ``heuristic" explanations as to how one 
produces, and avoids, skewness and/or minimizes or maximizes 
diffusion enhancement and accounts for various related factors 
(area of cross section, net flux) which can persist even after 
non-dimensionalization purely due to how one designs the non-dimensional 
object.

Explicitly,
\begin{enumerate}
\item Build/use finite element code on a platform that also allows for 
implementing a Monte Carlo style simulation
\item Quick validation of the code comparing asymptotic hallmarks 
(circle and parallel plates diffusivity enhancement and skewness).
\item do a full param sweep of the family of trapezoidal domains numerically. 
\item Do a numerical study of the param sweep, investigating
Zero, or near-zero level sets for geometric (ballistic) skewness 
as well as the asymptotic long time skewness coefficient
\item Given a sample of diffusion enhancement of the samples, 
implement a constrained optimization scheme to minimize 
diffusion enhancement keeping a fixed net flux; or fixed net 
cross sectional area; within reasonable bounds.
\item Try to implement ``learning" schemes by which computer-calculated 
asymptotics inform search directions for full numerical simulations.
\item Build evidence towards skewness associated to some notion of 
\textbf{girth} of the domain; maximum distance over all continuous 
paths in the domain connected to the center. 
\end{enumerate}

\section{Setup}
\subsection{Fluid and tracer equations}
Steady laminar fluid flow is driven by a constant pressure gradient. 
By convention, we choose the constant to be $-2$ so that the flow 
equation for a one-dimensional channel is $1-y^2$.

I choose $\Omega$ to represent the cross-section of the domain, and 
$\partial \Omega$ to be its boundary. The $x$-coordinate represents the 
flow direction, so $z$ and $y$ are coordinates for the cross-section. 
$y$ remains the vertical coordinate in this scheme, so $z$ is the horizontal coordinate. 
Notions of aspect ratio are represented by $\lambda$, and domains are 
aligned so that their axes are along the $z$ and $y$ directions. 
My meaning of aspect ratio is the ratio of short to long sides of the 
shape, so that $0 < \lambda \leq 1$, and a limit $\lambda \to 0$ for 
the rectengular cross sections is expected to reduce to the one-dimensional 
channel.
%
\begin{equation}
\label{flow_equation}
\Delta \tilde{u} = -2 \quad \left. \tilde{u} = 0 \right|_{\partial \Omega}
\end{equation}

\begin{equation}
\langle f \rangle \equiv \frac{\int_{\Omega} f dydz}{\int_{\Omega} 1 dydz}
\end{equation}

\begin{equation}
u \equiv \tilde{u} - \langle \tilde{u} \rangle
\end{equation}

\begin{equation}
C_t + \mathrm{Pe} \, u C_x = \Delta C, \quad \left. \underline{n} \cdot \nabla C\right|_{\partial \Omega} = 0 
\end{equation}

\subsection{Numerical tools}
Python 3.8.10 (cite) is used throughout this project. Numpy (cite) and 
matplotlib (cite)  are used throughout for general scientific 
computing and visualization.

The finite element code \texttt{fipy} was used as a framework for 
building meshes and accessing edges, cells, normals, boundary edges, etc. 
I implemented additional tools to simplify the interface to create 
a boundary by naming a list of $x$ and $y$ coordinates in two dimensions 
(For example: the fipy-level interface requires constructing a string to, in turn, call Gmsh.) 
An appeal of \texttt{fipy} is its support for expressing 
and solving partial differential equations (PDEs) without much code
form; for instance, one defines an object associated with 
equation (\ref{flow_equation}) as an object, then specifies which mesh 
edges represent boundary conditions (and their values) and then runs a 
\texttt{.solve()} method. This made calculations for asymptotic 
predictions fairly straightforward.

\subsection{Convention for domains}
In the general case, a domain is a polygon with an arbitrary 
number of points ${(x_i, y_i), \; i=1,\ldots,N }$ where 
consecutive points connect with line segments, and point $N$ connects 
with point $1$. The choice $N$ is only decided by the questions 
one wants to answer. The first part of this paper studies trapezoidal 
domains of a certain class. Because of the multiple symmetries, 
there will be only two free parameters instead of eight.

I define the family of trapezoidal domains using parameters $\lambda$ 
to represent aspect ratio, and $q$ to represent the ``eccentricity" 
(what's the proper word?). Trapezoid height is always $2$ by convention, 
and longer width is $2/\lambda$. The shorter width is $2q/\lambda$. 
Then, trapezoids are defined by a pair $(\lambda,q)$, where boundaries 
reduce to other shapes;
%
\begin{itemize}
\item Shapes $(\lambda,0)$ correspond to isosceles triangles whose base is 
$2/\lambda$ and height $2$;
\item Shapes $(\lambda,1)$ correspond to rectangles of aspect ratio $\lambda$;
\item I guess that's it.
\end{itemize}
%
A general challenge in shape optimization is a search space which 
grows exponentially with the number of points used to define the 
boundary. Here, a full parameter sweep is still feasible, 
which I will show below.

\section{Results}
idk lol
\begin{itemize}
\item Geometric and long-time skewness have unexpected nonlinear 
structure in $(\lambda,q)$ space.
\item Defining an arbitrary notion of ``skewness tolerance," a 
regions of this shape space can be identified which can control 
tracer skewness at short times (physically: ballistic problems) 
or long times (physically: diffusive problems); or both. Note these are 
\textbf{asymptotic predictions}; I am not aware of any theory 
which allows one to talk about how far skewness may deviate 
on intermediate times if short- and long-time asymptotics are 
given. Looking at our parameter sweeps in previous papers, 
this doesn't seem necessarily obvious unless those bounds are 
near zero.
\item 
\end{itemize}

\section{Conclusion}
\lipsum[1]

\end{document}
