\documentclass{amsart}

\usepackage[margin=1in]{geometry}
\usepackage{lipsum}
\usepackage{graphicx}

\title{Shape optimization for tracer distribution in Poiseuille flow}

\author{Manuchehr Aminian \\
California State Polytechnic University, Pomona \\ {maminian@cpp.edu}}



\begin{document}

\maketitle

\section{Motivation}
\begin{itemize}
\item Taylor dispersion applies to circular cross sections
\item Aris papers establish moment equations and solvability (?)
\item Chatwin studies cross section dependence of enhanced diffusivity 
in rectangles
\item George paper observes differing skewness in parallel plate 
and circular cross sections; what gives
\item Stone (?) and others explore cross section dependence and 
use shallow-water-like methods to reduce equations and arrive 
at approximate statements for some piecewise-defined domains
\item Our previous papers establish 
\begin{enumerate}
\item rectangular dependence and 
purely ballistic induced skewness with plug initial condition; (PRL)
\item Short and long time asymptotics; agreement with experiment; 
extension to ellipses (Science)
\item Further exploration of cross-section dependence by heuristic 
``racetrack" approach (SAPM)
\end{enumerate}
\end{itemize}

Given all these, my interest now is in a purely numerical exploration 
of cross section dependence which culminates from our previous work. 
I would like to arrive at ``heuristic" explanations as to how one 
produces, and avoids, skewness and/or minimizes or maximizes 
diffusion enhancement and accounts for various related factors 
(area of cross section, net flux) which can persist even after 
non-dimensionalization purely due to how one designs the non-dimensional 
object.

Explicitly,
\begin{enumerate}
\item \textbf{(loosely done; needs optimizaiton for MC)}
Build/use finite element code on a platform that also allows for 
implementing a Monte Carlo style simulation.
\item Validate finite element code convergence on equilateral triangle flow problem.
\item Quick validation of the code comparing asymptotic hallmarks 
(circle and parallel plates diffusivity enhancement and skewness).
\item do a full param sweep of the family of trapezoidal domains numerically. 
\item \textbf{(loosely done, pending tweaks to param sweep)} 
Do a numerical study of the param sweep, investigating
Zero, or near-zero level sets for geometric (ballistic) skewness 
as well as the asymptotic long time skewness coefficient
\item Given a sample of diffusion enhancement of the samples, 
implement a constrained optimization scheme to minimize 
diffusion enhancement keeping a fixed net flux; or fixed net 
cross sectional area; within reasonable bounds.
\item \textbf{(feeling less excited about this tonight)} Try to implement ``learning" schemes by which computer-calculated 
asymptotics inform search directions for full numerical simulations.
\item Build evidence towards skewness associated to some notion of 
\textbf{girth} of the domain; maximum distance over all continuous 
paths in the domain connected to the ``center" (peak flow point(s)).
\end{enumerate}

\section{Setup}
\subsection{Fluid and tracer equations}
First I will describe the general scenario and introduce notation. 

A passive tracer is assumed contained within a pipe or rectangular channel, 
advected by an ambient fluid flow. The solution of the fluid flow is 
not the main purpose for this study, but I describe the main steps here, as its 
solution enters into both asymptotic formulas as well as numerical simulation 
for the tracer. 

The flow itself is highly idealized, which allows for downstream analysis 
of behavior of the tracer. 
Fluid flow occurs only in the $\mathbf{x}$ direction of the pipe of 
infinite length in the $\mathbf{x}$ direction, with a fixed cross section 
independent of $x$. The 
perpendicular directions are denoted $\mathbf{z}$ and $\mathbf{y}$, and 
coordinates $(z,y)$. 
I refer to the cross section of the pipe or channel in either 
$y$ or $(y,z)$ as $\Omega$, 
and the boundary of this cross section $\partial \Omega$. 

The fluid flow is driven by a 
constant pressure gradient in the $\mathbf{x}$ direction, with no 
pressure gradient in the other directions, and is assumed in 
steady-state. The fluid itself 
is assumed incompressible with constant density. 
These assumptions constrain the flow sufficiently to a form $\mathbf{u} = (u(y,z), 0, 0)$. 
Further, I assume a no-slip boundary condition for the flow on the boundary of the pipe. 
All these taken together, the remaining equation for $u(y,z)$ is:
%
\begin{equation}
\Delta u(y,z) = -\frac{1}{\rho_0} \frac{\partial p}{\partial x}, \quad u(y,z)|_{\partial \Omega} = 0.
\end{equation}
%
A reasonable choice of nondimensionalization results in 
the right-hand side being replaced with $-2$. This choice is made following a convention 
of myself and my colleagues, which leads to a minimalist's 
solution for a one-dimensional channel (parallel infinite plates):
%
\begin{equation}
u_{yy} = -2 \quad \Rightarrow \quad u(y) = 1-y^2.
\end{equation}
%
My main interest in this paper is continuing investigation into influence of 
the shape of the cross section $\Omega$ on properties of the tracer distribution. 
As shown in prior work, a distinctly important property is aspect ratio. 
Aspect ratio in this paper is represented by $\lambda$, the ratio of the height (the span 
of the domain in $y$) to 
width (the span onf the domain in $z$) values in the cross section, 
so that $0 < \lambda \leq 1$, and a 
the behavior of limit $\lambda \to 0$ for 
the rectangular cross sections is expected to reduce to the one-dimensional 
channel.
Domains are generally 
aligned so that their principal axes (in a informal sense of the word) are 
along the $z$ and $y$ directions. 
%

To calculate of established asymptotic formulas for statistics, an averaging 
operator is needed. For arbitrary $f(y,z)$ and domain $\Omega$, I use angle brackets:
\begin{equation}
\langle f \rangle \equiv \frac{\int_{\Omega} f dzdy}{\int_{\Omega} 1 dzdy}.
\end{equation}

An early step in a full derivation of enhanced diffusivity 
is to center the analysis in the frame of reference of the 
mean speed of the flow by a Galiean transform $x \to x - vt$ for constant velocity $v$. 
Surprisingly, some asymptotic formulas rely on both 
calculations of the lab-frame and mean-zero flow, so notation for both is needed. 
Since working with the mean-zero flow is more common, 
I denote the lab-frame flow $\tilde{u}$ and mean-zero flow $u$, which are related to 
each other by:
%
\begin{equation}
u = \tilde{u} - \langle \tilde{u} \rangle.
\end{equation}
%
The constant velocity term for the Galilean transform in this context 
is $v = \langle \tilde{u} \rangle$.

Finally, the advection-diffusion equation for the tracer $C(x,y,z,t)$ is written 
as: 
\begin{equation}
C_t + \mathrm{Pe} \, u(y,z) C_x = \Delta C, \quad \left. \underline{n} \cdot \nabla C\right|_{\partial \Omega} = 0.
\end{equation}
Zero Neumann boundary conditions are assumed which represent a no-flux condition 
for the tracer.

\subsection{Numerical tools}
Python 3.8.10 (cite) is used throughout this project. Numpy (cite) and 
matplotlib (cite)  are used throughout for general scientific 
computing and visualization.

The finite element code \texttt{fipy} was used as a framework for 
building meshes and accessing edges, cells, normals, boundary edges, etc. 
I implemented additional tools to simplify the creation of 
a domain through its boundary by naming a list of $x$ and $y$ coordinates in 
two dimensions. 
Note the fipy-level interface requires programmatically constructing a string to, 
in turn, call the meshing library \texttt{Gmsh}. 
The value of \texttt{fipy} is apparent in its support for expressing 
and solving partial differential equations without much technical coding baggage. 
For instance, to get a finite element solution for the flow, 
one defines an object associated with 
equation (\ref{flow_equation}), where Laplace operator is specified abstractly, 
then specifies which elements of the mesh 
represent boundary conditions (and their values), then runs a 
\texttt{.solve()} method. 

\subsection{Conventions for arbitrary domains}
In the general case, a domain is a polygon with an arbitrary 
number of points ${(z_i, y_i), \; i=1,\ldots,N }$ where 
consecutive points connect with line segments, and point $N$ by convention connects 
back to point $1$. The choice of number of points $N$ is only decided by the questions 
one wants to answer. The first part of this paper studies trapezoidal 
domains of a certain class, where $N=4$. Because of the multiple symmetries and 
invariants, there is in fact only two free parameters instead of eight ($2N$).

The family of trapezoids I work with are illustrated in \ref{fig:TRAP_SCHEMATIC}. 
The longer width is $2/\lambda$, understood as an aspect ratio. A 
second parameter $q$ (``eccentricity") defines the shorter width, $2q/\lambda$, 
relative to the aspect ratio. The trapezoid heights are chosen as $2$ by convention. 
%
\begin{figure}
\includegraphics[width=\linewidth]{images2/trapezoid_param_schematic.pdf}
\caption{General schematic of trapezoid domains with a few examples illustrated. 
Panel A: Choices of $\lambda$ affect the ratio of height to width of the longer side. 
Choices of $q$ change the ratio of the shorter to longer widths of the trapezoid. 
Panel B, C: Resulting trapezoids for two other pairs of $(\lambda,q)$ shown.}
\label{fig:TRAP_SCHEMATIC}
\end{figure}
%
Then, these trapezoids are fully defined by a pair $(\lambda,q)$. 
Specific values of $(\lambda,q)$ reduce to other shapes:
%
\begin{itemize}
\item Shapes $(\lambda,0)$ correspond to isosceles triangles whose base is 
$2/\lambda$ and height is $2$.
\item In particular, a bit of trigonometry shows 
$(\sqrt{3}/2, 0)$ produces to an equilateral triangle which 
has an exact flow solution, which can be useful for validating convergence of 
the finite element solvers.
\item Shapes $(\lambda,1)$ correspond to rectangles of aspect ratio $\lambda$.
\end{itemize}
%
While the final goal of this paper is an unbiased search over a space of cross sections, 
which is challenging class of problems because of an exponentially growing 
search space in the degrees of freedom, these class of trapezoids are still 
tractable for a full parameter sweep, which I investigate first.

\section{Results}
idk lol

\begin{figure}
\includegraphics[width=0.6\linewidth]{images/short_long_intersection.pdf}
\end{figure}

\begin{itemize}
\item Geometric and long-time skewness have unexpected nonlinear 
structure in $(\lambda,q)$ space.
\item Defining an arbitrary notion of ``skewness tolerance," a 
regions of this shape space can be identified which can control 
tracer skewness at short times (physically: ballistic problems) 
or long times (physically: diffusive problems); or both. Note these are 
\textbf{asymptotic predictions}; I am not aware of any theory 
which allows one to talk about how far skewness may deviate 
on intermediate times if short- and long-time asymptotics are 
given. Looking at our parameter sweeps in previous papers, 
this doesn't seem necessarily obvious unless those bounds are 
near zero.
\item 
\end{itemize}

\section{Conclusion}
\lipsum[1]

\end{document}
